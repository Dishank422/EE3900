\documentclass[9pt]{beamer}
\usepackage[utf8]{inputenc}
\usepackage{listings}
\usepackage{amsmath}
\usepackage{graphicx}
\usepackage{tikz}
\usepackage{pgfplots}
\usepgfplotslibrary{external}
\tikzexternalize
\pgfplotsset{compat = 1.9}
\usetheme{Boadilla}
\lstset{
%language=C,
frame=single, 
breaklines=true,
columns=fullflexible
}
\title{Gate Assignment 1}
\author{Dishank - AI20BTECH11011}
\date{}

\begin{document}
\newcommand{\myvec}[1]{\ensuremath{\begin{pmatrix}#1\end{pmatrix}}}


\begin{frame}
\titlepage
\end{frame}

\begin{frame}
\frametitle{Question}
Let the state-space representation on an LTI system be $\dot{x}(t) = Ax(t)+Bu(t)$, $y(t)=Cx(t)+du(t)$ where A,B,C are matrices,  d is a scalar, u(t) is the input to the system, and y(t) is its output. Let $B = \myvec{ 0 & 0 &  1}^\top$ and $d = 0$. Which one of the following options for A and C will ensure that the transfer function of this LTI system is 

\begin{equation}
    H(s) = \dfrac{1}{s^3+3s^2+2s+1}
\end{equation}

\begin{enumerate}[A]
    \item $A = \myvec{
     0 &  1 &  0\\ 
     0 &  0 &  1\\
    -1 & -2 & -3
    }$ and $C = \myvec{1 & 0 & 0}$
    \item $A = \myvec{
     0 &  1 &  0\\ 
     0 &  0 &  1\\
    -3 & -2 & -1
    }$ and $C = \myvec{1 & 0 & 0}$
    \item $A = \myvec{
     0 &  1 &  0\\ 
     0 &  0 &  1\\
    -1 & -2 & -3
    }$ and $C = \myvec{0 & 0 & 1}$
    \item $A = \myvec{
     0 &  1 &  0\\ 
     0 &  0 &  1\\
    -3 & -2 & -1
    }$ and $C = \myvec{0 & 0 & 1}$
\end{enumerate}
\end{frame}

\begin{frame}{Laplace transform}
    
    \begin{equation}
        Assumption: \; x(t) = 0\; at\; t=0
    \end{equation}
    \begin{align}
        \mathcal{L}(x(t)) = X(s) &= \int_{0}^{\infty}x(t)e^{-st}dt\\
        \implies \mathcal{L}(\dot{x}(t)) &= \int_{0}^{\infty}\dot{x}(t)e^{-st}dt\\
                                         &= e^{-st}x(t)\big \lvert_{0}^{\infty} + \int_{0}^{\infty}sx(t)e^{-st}dt\\
                                         &= -x(0)+sX(s)\\
                                         &= sX(s)
    \end{align}
\end{frame}

\begin{frame}{Solution}
    We are given 
\begin{equation}
    \myvec{\dot{x}(t)\\y(t)} = \myvec{A & B\\C & d}\myvec{x(t)\\u(t)}
\end{equation}
    
Taking Laplace transform on both sides,
\begin{align}
    \myvec{sX(s)\\Y(s)} &= \myvec{A & B\\ C & d}\myvec{X(s)\\U(s)}\\
    \implies sX(s) &= AX(s)+BU(s)\\
    \implies X(s) &= (sI-A)^{-1} BU(s)\\
    \implies Y(s) &= CX(s)+dU(s)\\ 
                  &= C(sI-A)^{-1} BU(s) +dU(s)
\end{align}
By definition, 
\begin{align}
    Y(s) &= H(s)U(s)\\
    \implies H(s) &= C(sI-A)^{-1} B + d\\
                  &= \dfrac{1}{s^3+3s^2+2s+1}
\end{align} 
\begin{equation}
    \implies C(sI-A)^{-1} B + d = \dfrac{1}{s^3+3s^2+2s+1}\label{result}
\end{equation}

\end{frame}

\begin{frame}{Solution Contd.}
    Now we substitute A and C from each option into eq. \ref{result} and verify if H(s) is the same as we require. 
\begin{enumerate}[A]
    \item \begin{equation}
        C(sI-A)^{-1} B +d = \dfrac{1}{s^3+3s^2+2s+1}
    \end{equation}
    \item \begin{equation}
        C(sI-A)^{-1} B +d = \dfrac{1}{s^3+1s^2+2s+3}
    \end{equation}    
    \item \begin{equation}
        C(sI-A)^{-1} B +d = \dfrac{s^2}{s^3+3s^2+2s+1}
    \end{equation}
    \item \begin{equation}
        C(sI-A)^{-1} B +d = \dfrac{s^2}{s^3+1s^2+2s+3}
    \end{equation}
\end{enumerate}

Hence only option A is the correct option. For above calculations, refer https://github.com/Dishank422/EE3900/blob/main/Gate-Assignment1/codes/codes.py
\end{frame}

\end{document}

