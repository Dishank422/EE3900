\documentclass[journal,12pt,twocolumn]{IEEEtran}

\usepackage{setspace}
\usepackage{gensymb}

\singlespacing


\usepackage[cmex10]{amsmath}

\usepackage{amsthm}

\usepackage{mathrsfs}
\usepackage{txfonts}
\usepackage{stfloats}
\usepackage{bm}
\usepackage{cite}
\usepackage{cases}
\usepackage{subfig}

\usepackage{longtable}
\usepackage{multirow}

\usepackage{enumitem}
\usepackage{mathtools}
\usepackage{steinmetz}
\usepackage{tikz}
\usepackage{circuitikz}
\usepackage{verbatim}
\usepackage{tfrupee}
\usepackage[breaklinks=true]{hyperref}
\usepackage{graphicx}
\usepackage{tkz-euclide}
\usepackage{float}

\usetikzlibrary{calc,math}
\usepackage{listings}
    \usepackage{color}                                            %%
    \usepackage{array}                                            %%
    \usepackage{longtable}                                        %%
    \usepackage{calc}                                             %%
    \usepackage{multirow}                                         %%
    \usepackage{hhline}                                           %%
    \usepackage{ifthen}                                           %%
    \usepackage{lscape}     
\usepackage{multicol}
\usepackage{chngcntr}

\DeclareMathOperator*{\Res}{Res}

\renewcommand\thesection{\arabic{section}}
\renewcommand\thesubsection{\thesection.\arabic{subsection}}
\renewcommand\thesubsubsection{\thesubsection.\arabic{subsubsection}}

\renewcommand\thesectiondis{\arabic{section}}
\renewcommand\thesubsectiondis{\thesectiondis.\arabic{subsection}}
\renewcommand\thesubsubsectiondis{\thesubsectiondis.\arabic{subsubsection}}


\hyphenation{op-tical net-works semi-conduc-tor}
\def\inputGnumericTable{}                                 %%

\lstset{
%language=C,
frame=single, 
breaklines=true,
columns=fullflexible
}
\begin{document}
\newtheorem{theorem}{Theorem}[section]
\newtheorem{problem}{Problem}
\newtheorem{proposition}{Proposition}[section]
\newtheorem{lemma}{Lemma}[section]
\newtheorem{corollary}[theorem]{Corollary}
\newtheorem{example}{Example}[section]
\newtheorem{definition}[problem]{Definition}

\newcommand{\BEQA}{\begin{eqnarray}}
\newcommand{\EEQA}{\end{eqnarray}}
\newcommand{\define}{\stackrel{\triangle}{=}}
\bibliographystyle{IEEEtran}
\providecommand{\mbf}{\mathbf}
\providecommand{\pr}[1]{\ensuremath{\Pr\left(#1\right)}}
\providecommand{\qfunc}[1]{\ensuremath{Q\left(#1\right)}}
\providecommand{\sbrak}[1]{\ensuremath{{}\left[#1\right]}}
\providecommand{\lsbrak}[1]{\ensuremath{{}\left[#1\right.}}
\providecommand{\rsbrak}[1]{\ensuremath{{}\left.#1\right]}}
\providecommand{\brak}[1]{\ensuremath{\left(#1\right)}}
\providecommand{\lbrak}[1]{\ensuremath{\left(#1\right.}}
\providecommand{\rbrak}[1]{\ensuremath{\left.#1\right)}}
\providecommand{\cbrak}[1]{\ensuremath{\left\{#1\right\}}}
\providecommand{\lcbrak}[1]{\ensuremath{\left\{#1\right.}}
\providecommand{\rcbrak}[1]{\ensuremath{\left.#1\right\}}}
\theoremstyle{remark}
\newtheorem{rem}{Remark}
\newcommand{\sgn}{\mathop{\mathrm{sgn}}}
\providecommand{\abs}[1]{\vert#1\vert}
\providecommand{\res}[1]{\Res\displaylimits_{#1}} 
\providecommand{\norm}[1]{\lVert#1\rVert}
%\providecommand{\norm}[1]{\lVert#1\rVert}
\providecommand{\mtx}[1]{\mathbf{#1}}
\providecommand{\mean}[1]{E[ #1 ]}
\providecommand{\fourier}{\overset{\mathcal{F}}{ \rightleftharpoons}}
%\providecommand{\hilbert}{\overset{\mathcal{H}}{ \rightleftharpoons}}
\providecommand{\system}{\overset{\mathcal{H}}{ \longleftrightarrow}}
	%\newcommand{\solution}[2]{\textbf{Solution:}{#1}}
\newcommand{\solution}{\noindent \textbf{Solution: }}
\newcommand{\cosec}{\,\text{cosec}\,}
\providecommand{\dec}[2]{\ensuremath{\overset{#1}{\underset{#2}{\gtrless}}}}
\newcommand{\myvec}[1]{\ensuremath{\begin{pmatrix}#1\end{pmatrix}}}
\newcommand{\mydet}[1]{\ensuremath{\begin{vmatrix}#1\end{vmatrix}}}
\numberwithin{equation}{subsection}
\makeatletter
\@addtoreset{figure}{problem}
\makeatother
\let\StandardTheFigure\thefigure
\let\vec\mathbf
\renewcommand{\thefigure}{\theproblem}
\def\putbox#1#2#3{\makebox[0in][l]{\makebox[#1][l]{}\raisebox{\baselineskip}[0in][0in]{\raisebox{#2}[0in][0in]{#3}}}}
     \def\rightbox#1{\makebox[0in][r]{#1}}
     \def\centbox#1{\makebox[0in]{#1}}
     \def\topbox#1{\raisebox{-\baselineskip}[0in][0in]{#1}}
     \def\midbox#1{\raisebox{-0.5\baselineskip}[0in][0in]{#1}}
\vspace{3cm}
\title{ASSIGNMENT 1}
\author{Dishank Jain \\ AI20BTECH11011}
\maketitle
\newpage
\bigskip
\renewcommand{\thefigure}{\theenumi}
\renewcommand{\thetable}{\theenumi}
Download all python codes from 
%
\begin{lstlisting}
https://github.com/Dishank422/EE3900/blob/main/Gate-Assignment1/codes
\end{lstlisting}
%
and latex-tikz codes from 
%
\begin{lstlisting}
https://github.com/Dishank422/EE3900/blob/main/Gate-Assignment1/latex_code.tex
\end{lstlisting}
%
\section{EC 2019 Q.33}
Let the state-space representation on an LTI system be $\dot{x}(t) = Ax(t)+Bu(t)$, $y(t)=Cx(t)+du(t)$ where A,B,C are matrices,  d is a scalar, u(t) is the input to the system, and y(t) is its output. Let $B = \myvec{ 0 & 0 &  1}^\top$ and $d = 0$. Which one of the following options for A and C will ensure that the transfer function of this LTI system is 
\begin{equation}
    H(s) = \dfrac{1}{s^3+3s^2+2s+1}
\end{equation}

\begin{enumerate}[label = (\Alph*)]
    \item $A = \myvec{
     0 &  1 &  0\\ 
     0 &  0 &  1\\
    -1 & -2 & -3
    }$ and $C = \myvec{1 & 0 & 0}$
    \item $A = \myvec{
     0 &  1 &  0\\ 
     0 &  0 &  1\\
    -3 & -2 & -1
    }$ and $C = \myvec{1 & 0 & 0}$
    \item $A = \myvec{
     0 &  1 &  0\\ 
     0 &  0 &  1\\
    -1 & -2 & -3
    }$ and $C = \myvec{0 & 0 & 1}$
    \item $A = \myvec{
     0 &  1 &  0\\ 
     0 &  0 &  1\\
    -3 & -2 & -1
    }$ and $C = \myvec{0 & 0 & 1}$
\end{enumerate}

\section{Solution}

We are given 
\begin{equation}
    \myvec{\dot{x}(t)\\y(t)} = \myvec{A & B\\C & d}\myvec{x(t)\\u(t)}
\end{equation}
    
Taking Laplace transform on both sides,
\begin{align}
    \myvec{sX(s)\\Y(s)} &= \myvec{A & B\\ C & d}\myvec{X(s)\\U(s)}\\
    \implies sX(s) &= AX(s)+BU(s)\\
    \implies X(s) &= (sI-A)^{-1} BU(s)\\
    \implies Y(s) &= CX(s)+dU(s)\\ 
                  &= C(sI-A)^{-1} BU(s) +dU(s)
\end{align}

By definition, 
\begin{align}
    Y(s) &= H(s)U(s)\\
    \implies H(s) &= C(sI-A)^{-1} B + d\\
                  &= \dfrac{1}{s^3+3s^2+2s+1}
\end{align} 
\begin{equation}
    \implies C(sI-A)^{-1} B + d = \dfrac{1}{s^3+3s^2+2s+1}\label{result}
\end{equation}

Now we cross verify the options with eq \ref{result}. Using the codes from codes/codes.py,
\begin{enumerate}[label = (\Alph*)]
    \item \begin{equation}
        C(sI-A)^{-1} B +d = \dfrac{1}{s^3+3s^2+2s+1}
    \end{equation}
    \item \begin{equation}
        C(sI-A)^{-1} B +d = \dfrac{1}{s^3+1s^2+2s+3}
    \end{equation}    
    \item \begin{equation}
        C(sI-A)^{-1} B +d = \dfrac{s^2}{s^3+3s^2+2s+1}
    \end{equation}
    \item \begin{equation}
        C(sI-A)^{-1} B +d = \dfrac{s^2}{s^3+1s^2+2s+3}
    \end{equation}
\end{enumerate}

Hence only option A is the correct option.
\end{document}
