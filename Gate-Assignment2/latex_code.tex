\documentclass[journal,12pt,twocolumn]{IEEEtran}

\usepackage{setspace}
\usepackage{gensymb}

\singlespacing


\usepackage[cmex10]{amsmath}

\usepackage{amsthm}

\usepackage{mathrsfs}
\usepackage{txfonts}
\usepackage{stfloats}
\usepackage{bm}
\usepackage{cite}
\usepackage{cases}
\usepackage{subfig}

\usepackage{longtable}
\usepackage{multirow}

\usepackage{enumitem}
\usepackage{mathtools}
\usepackage{steinmetz}
\usepackage{tikz}
\usepackage{circuitikz}
\usepackage{verbatim}
\usepackage{tfrupee}
\usepackage[breaklinks=true]{hyperref}
\usepackage{graphicx}
\usepackage{tkz-euclide}
\usepackage{float}

\usetikzlibrary{calc,math}
\usepackage{listings}
    \usepackage{color}                                            %%
    \usepackage{array}                                            %%
    \usepackage{longtable}                                        %%
    \usepackage{calc}                                             %%
    \usepackage{multirow}                                         %%
    \usepackage{hhline}                                           %%
    \usepackage{ifthen}                                           %%
    \usepackage{lscape}     
\usepackage{multicol}
\usepackage{chngcntr}

\DeclareMathOperator*{\Res}{Res}

\renewcommand\thesection{\arabic{section}}
\renewcommand\thesubsection{\thesection.\arabic{subsection}}
\renewcommand\thesubsubsection{\thesubsection.\arabic{subsubsection}}

\renewcommand\thesectiondis{\arabic{section}}
\renewcommand\thesubsectiondis{\thesectiondis.\arabic{subsection}}
\renewcommand\thesubsubsectiondis{\thesubsectiondis.\arabic{subsubsection}}


\hyphenation{op-tical net-works semi-conduc-tor}
\def\inputGnumericTable{}                                 %%

\lstset{
%language=C,
frame=single, 
breaklines=true,
columns=fullflexible
}
\begin{document}
\newtheorem{theorem}{Theorem}[section]
\newtheorem{problem}{Problem}
\newtheorem{proposition}{Proposition}[section]
\newtheorem{lemma}{Lemma}[section]
\newtheorem{corollary}[theorem]{Corollary}
\newtheorem{example}{Example}[section]
\newtheorem{definition}[problem]{Definition}

\newcommand{\BEQA}{\begin{eqnarray}}
\newcommand{\EEQA}{\end{eqnarray}}
\newcommand{\define}{\stackrel{\triangle}{=}}
\bibliographystyle{IEEEtran}
\providecommand{\mbf}{\mathbf}
\providecommand{\pr}[1]{\ensuremath{\Pr\left(#1\right)}}
\providecommand{\qfunc}[1]{\ensuremath{Q\left(#1\right)}}
\providecommand{\sbrak}[1]{\ensuremath{{}\left[#1\right]}}
\providecommand{\lsbrak}[1]{\ensuremath{{}\left[#1\right.}}
\providecommand{\rsbrak}[1]{\ensuremath{{}\left.#1\right]}}
\providecommand{\brak}[1]{\ensuremath{\left(#1\right)}}
\providecommand{\lbrak}[1]{\ensuremath{\left(#1\right.}}
\providecommand{\rbrak}[1]{\ensuremath{\left.#1\right)}}
\providecommand{\cbrak}[1]{\ensuremath{\left\{#1\right\}}}
\providecommand{\lcbrak}[1]{\ensuremath{\left\{#1\right.}}
\providecommand{\rcbrak}[1]{\ensuremath{\left.#1\right\}}}
\theoremstyle{remark}
\newtheorem{rem}{Remark}
\newcommand{\sgn}{\mathop{\mathrm{sgn}}}
\providecommand{\abs}[1]{\vert#1\vert}
\providecommand{\res}[1]{\Res\displaylimits_{#1}} 
\providecommand{\norm}[1]{\lVert#1\rVert}
%\providecommand{\norm}[1]{\lVert#1\rVert}
\providecommand{\mtx}[1]{\mathbf{#1}}
\providecommand{\mean}[1]{E[ #1 ]}
\providecommand{\fourier}{\overset{\mathcal{F}}{ \rightleftharpoons}}
%\providecommand{\hilbert}{\overset{\mathcal{H}}{ \rightleftharpoons}}
\providecommand{\system}{\overset{\mathcal{H}}{ \longleftrightarrow}}
	%\newcommand{\solution}[2]{\textbf{Solution:}{#1}}
\newcommand{\solution}{\noindent \textbf{Solution: }}
\newcommand{\cosec}{\,\text{cosec}\,}
\providecommand{\dec}[2]{\ensuremath{\overset{#1}{\underset{#2}{\gtrless}}}}
\newcommand{\myvec}[1]{\ensuremath{\begin{pmatrix}#1\end{pmatrix}}}
\newcommand{\mydet}[1]{\ensuremath{\begin{vmatrix}#1\end{vmatrix}}}
\numberwithin{equation}{subsection}
\makeatletter
\@addtoreset{figure}{problem}
\makeatother
\let\StandardTheFigure\thefigure
\let\vec\mathbf
\renewcommand{\thefigure}{\theproblem}
\def\putbox#1#2#3{\makebox[0in][l]{\makebox[#1][l]{}\raisebox{\baselineskip}[0in][0in]{\raisebox{#2}[0in][0in]{#3}}}}
     \def\rightbox#1{\makebox[0in][r]{#1}}
     \def\centbox#1{\makebox[0in]{#1}}
     \def\topbox#1{\raisebox{-\baselineskip}[0in][0in]{#1}}
     \def\midbox#1{\raisebox{-0.5\baselineskip}[0in][0in]{#1}}
\vspace{3cm}
\title{GATE ASSIGNMENT 2}
\author{Dishank Jain \\ AI20BTECH11011}
\maketitle
\newpage
\bigskip
\renewcommand{\thefigure}{\theenumi}
\renewcommand{\thetable}{\theenumi}
Download all python codes from 
%
\begin{lstlisting}
https://github.com/Dishank422/EE3900/blob/main/Gate-Assignment2/codes
\end{lstlisting}
%
and latex-tikz codes from
%
\begin{lstlisting}
https://github.com/Dishank422/EE3900/blob/main/Gate-Assignment2/latex_code.tex
\end{lstlisting}
%
\section{EC 2019 Q.33}
The DFT of a vector $\myvec{a & b & c & d}$ is the vector $\myvec{\alpha & \beta & \gamma & \delta}$. Consider the product 
\begin{equation}
    \myvec{p & q & r & s} = \myvec{a & b & c & d}\myvec{a & b & c & d\\ d & a & b & c\\ c & d & a & b\\b & c & d &a}
\end{equation}
The DFT of the vector $\myvec{p & q & r & s}$ is a scaled version of 
\begin{enumerate}[label = (\Alph*)]
    \item $\myvec{\alpha^2 & \beta ^2 & \gamma^2 & \delta^2}$
    \item $\myvec{\sqrt{\alpha} & \sqrt{\beta} & \sqrt{\gamma} & \sqrt{\delta}}$
    \item $\myvec{\alpha+\beta & \beta+\delta & \delta+\gamma & \gamma+\alpha}$
    \item $\myvec{\alpha & \beta & \gamma & \delta}$
\end{enumerate}

\section{Solution}
\begin{lemma}
If $\vec{T}$ is a circulant matrix, then the eigenvector matrix of $\vec{T}$ is the same as the DFT matrix $\vec{W}$ and the eigenvalues are the DFT of the first column of $\vec{T}$.
\end{lemma}

\begin{proof}
The $i^{th}$ column of the $n\times n$ DFT matrix is given by
\begin{equation}
    p_i = \dfrac{1}{\sqrt{n}}\myvec{1 \\ \omega^{i} \\ \omega^{2i} \\ : \\ : \\ \omega^{(n-1)i}} 
\end{equation}
where $\omega$ is the $n^{th}$ root of 1. We shall show that this $p_i$ is the eigenvector of $\vec{T}$. Observe that the k$^{th}$ component of $\vec{T}\vec{p}_i$ is given by

\begin{align}
    y_k &= \dfrac{1}{\sqrt{n}}\sum_{j = 0}^{n-1}\vec{T}_{kj} \omega^{ij}\\ 
    &= \dfrac{\omega^{ki}}{\sqrt{n}}\sum_{j = 0}^{n-1}\vec{T}_{kj}\omega^{(j-k)i}\\
    &= \dfrac{\omega^{ki}}{\sqrt{n}}\sum_{j = 0}^{n-1}\vec{T}_{(j-k)mod(n)1}\omega^{(j-k)i}\\
    &= \dfrac{\omega^{ki}}{\sqrt{n}}\sum_{m = 0}^{n-1}\vec{T}_{m1}\omega^{mi}
\end{align}

Therefore 
\begin{equation}
    \vec{T}\vec{p}_i = \dfrac{\sum_{m = 0}^{n-1}\vec{T}_{m1}\omega^{mi}}{\sqrt{n}}\myvec{1 \\ \omega^{i} \\ \omega^{2i} \\ : \\ : \\ \omega^{(n-1)i}}
\end{equation}

But $\sum_{m = 0}^{n-1}\vec{T}_{m1}\omega^{mi}$ is nothing but the i$^{th}$ element of the DFT of the first column of $\vec{T}$. Therefore $\vec{p}_i$ is an eigenvector of $\vec{T}$ with eigenvalue as i$^{th}$ element of the DFT of the first column of $\vec{T}$.
\end{proof}
 
Now we start with the solution. First we express the equations in a more convenient form.
\begin{align}
    \myvec{p & q & r & s} &= \myvec{a & b & c & d}\myvec{a & b & c & d\\ d & a & b & c\\ c & d & a & b\\b & c & d & a}\\
    \implies \myvec{p & q & r & s}^\top &= \myvec{a & b & c & d\\ d & a & b & c\\ c & d & a & b\\b & c & d & a}^\top \myvec{a & b & c & d}^\top
\end{align}
\begin{equation}
    \implies \myvec{p \\ q \\ r \\ s} = \myvec{a & d & c & b\\ b & a & d & c\\ c & b & a & d\\d & c & b & a}\myvec{a \\ b \\ c \\ d}
\end{equation}

\begin{align}
    Let\; \vec{x} = &\myvec{a \\ b \\ c \\ d};\; \vec{X} = \myvec{\alpha \\ \beta \\ \gamma \\ \delta};\; \vec{y} = \myvec{p \\ q \\ r \\ s}\\
    &\vec{T} = \myvec{a & d & c & b\\ b & a & d & c\\ c & b & a & d\\d & c & b & a}
\end{align}
Then we have to find $\vec{Y}$ the DFT of $\vec{y}$. We know
\begin{align}
    &\vec{X} = \vec{W}\vec{x}\\
    \implies &\vec{x} = \vec{W}^{-1} \vec{X}\\
    &\vec{y} = \vec{T}\vec{x}\\
    &\vec{Y} = \vec{W}\vec{y}\\
    \implies &\vec{Y} = \vec{W}\vec{T}\vec{x}\\
    \implies &\vec{Y} = \vec{W}\vec{T}\vec{W}^{-1}\vec{X}
\end{align}

But $\vec{T}$ is a circulant matrix with first column as $\myvec{a \\ b \\ c \\ d}$. Therefore the eigenvalues are $\myvec{\alpha \\ \beta \\ \gamma \\ \delta}$. Using eigen decomposition
\begin{align}
    \vec{T} = \vec{W}&\myvec{\alpha & 0 & 0 & 0\\ 0 & \beta & 0 & 0\\ 0 & 0 & \gamma & 0\\ 0 & 0 & 0 & \delta} \vec{W}^{-1}\\
    \implies \vec{Y} = \vec{W}\vec{W}&\myvec{\alpha & 0 & 0 & 0\\ 0 & \beta & 0 & 0\\ 0 & 0 & \gamma & 0\\ 0 & 0 & 0 & \delta} \vec{W}^{-1}\vec{W}^{-1}\vec{X}
\end{align}
Using properties of DFT
\begin{align}
    \vec{W}\vec{W} &= \vec{I};\; \vec{W}^{-1}\vec{W}^{-1} = \vec{I}\\
    \implies \vec{Y} &= \myvec{\alpha & 0 & 0 & 0\\ 0 & \beta & 0 & 0\\ 0 & 0 & \gamma & 0\\ 0 & 0 & 0 & \delta} \vec{X}\\
    \implies \vec{Y} &= \myvec{\alpha^2 \\ \beta^2 \\ \gamma^2 \\ \delta^2}
\end{align}
Therefore option (A) is the correct option.
\end{document}
